\section{Introduction}

In the past decade, data has become ubiquitous, and as a result, the need to store large amounts of data in an efficient way has become increasingly important. Databases have emerged as the preferred method for organizing and storing information in an organized and efficient way. Since the 1970s, most database management systems have been designed around the relational model\cite{relational_model} where the most fundamental elements that characterize it are relations, commonly referred to as tables. A relation is a collection of tuples, or table rows, each of which shares a set of characteristics, or table columns. The most typical method of retrieving data from a DBMS is to send a query, or a structured request for a set of data. 

Let's suppose that we have access to a large number of queries that each user submits to a database management system (DBMS), each of which is associated with a rating that indicates how satisfied the user is with the query's outcome. In particular, let's assume that our DBMS is made up of a single relation that allows users to submit multiple queries to retrieve data. The question that remains is whether we can exploit all of this data to suggest queries to users based on previous interests. 

% Why you think that such a study is important? 
Making recommendations is essential in many fields, starting with e-commerce websites where these systems attempt to suggest the best products that match the user's interest in order to improve sales. This methodology is used in e-commerce as well as libraries, as in the case of the aforementioned Grundy system. Another scenario is when streaming services try to find the movie that aligns most with a user's preferences.

%Introduction to the recommendation system
These systems are usually based on the so-called \emph{Utility Matrix}, which captures a user's preference for a particular item offered by the service. The matrix itself has some blank spaces, as usually users don't have recorded data about each item of the system. So the goal of the recommendation systems is to provide meaningful values for the blank spaces inside the matrix.


% why it is challenging (i.e., not trivial) to perform this processing? What were the hard/challenging parts in developing a solution?
There is an infinite amount of ways to create a recommendation system, starting with the two fundamental approaches: content based and collaborative filtering. These are not the only ones; in fact, clustering is another option that can be taken into account while looking for groups of commonalities that can suggest one thing to another member of the same group. Each of these approaches must cope with the problem of dealing with huge amount of data and finding the best recommendation in the shortest amount of time; in most of the cases, it shouldn't take more than a few seconds to find a good recommendation, as is the case with e-commerce websites where the customer can see recommendations right away. In general, making recommendations based on massive volumes of data is still a challenging research problem. 

% introduction to databases and query and the problem statement
As recommendation systems are mainly based on already existing data, the presence of some sort of database is necessary for the workflow of the algorithm.
These databases are usually based on the relational model and thus organized in tables made of rows and columns, built with the purpose to have an easy architecture and more understandable data. This paper first demonstrate how locality sensitive hashing can be used to improve similar items search to provide user recommendations, and then shows how the incorporation of a content-based approach to build a hybrid recommendation system may be beneficial for the recommendations' accuracy. 
