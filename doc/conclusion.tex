\section{Conclusion}

The solutions presented and examined in this research produced satisfactory results, demonstrating how the integration of several data mining approaches may result in an advanced recommendation system that can make meaningful recommendations. The final algorithm was built one block at a time, producing two distinct algorithms: the first one based on collaborative filtering with LSH, which keeps both a high accuracy and good time performance, and a second solution, which integrates content based to the previous solution to build an hybrid recommendation system with an even greater accuracy but worse time performance. The experiments conducted in Section \ref{experimental_evalutation} demonstrate how combining a content-based approach with the a collaborative filtering method results in a hybrid recommendation system that generates recommendations that are more accurate. Furthermore it was also possible to demonstrate that standard methodologies, such as the Naive algorithm using only collaborative filtering of Section \ref{naive_solution}, are inapplicable in the case of large datasets, but that combining various strategies, as seen in LSH, can result in solutions that enable dealing with massive datasets. To sum up, this research made it possible to discover that the problem of providing useful query recommendations in the context of DBMS is a very challenging task that depends on a wide range of factors. 


In conclusion, the research presented in this paper has demonstrated the effectiveness of the proposed query recommendation system in addressing the specific problem addressed. However, as with any new system, there is still much work to be done in terms of implementation and fine-tuning. In particular, future work could focus on implementing the solution using the map-reduce framework: this would allow for more efficient and scalable processing of large datasets, which is crucial for the success of query recommendation systems. Additionally, it would be valuable to investigate the possibility of integrating the proposed solution with other technologies to enhance its overall performance